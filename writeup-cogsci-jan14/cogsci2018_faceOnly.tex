% Template for Cogsci submission with R Markdown

% Stuff changed from original Markdown PLOS Template
\documentclass[10pt, letterpaper]{article}

\usepackage{cogsci}
\usepackage{pslatex}
\usepackage{float}
\usepackage{caption}

% amsmath package, useful for mathematical formulas
\usepackage{amsmath}

% amssymb package, useful for mathematical symbols
\usepackage{amssymb}

% hyperref package, useful for hyperlinks
\usepackage{hyperref}

% graphicx package, useful for including eps and pdf graphics
% include graphics with the command \includegraphics
\usepackage{graphicx}

% Sweave(-like)
\usepackage{fancyvrb}
\DefineVerbatimEnvironment{Sinput}{Verbatim}{fontshape=sl}
\DefineVerbatimEnvironment{Soutput}{Verbatim}{}
\DefineVerbatimEnvironment{Scode}{Verbatim}{fontshape=sl}
\newenvironment{Schunk}{}{}
\DefineVerbatimEnvironment{Code}{Verbatim}{}
\DefineVerbatimEnvironment{CodeInput}{Verbatim}{fontshape=sl}
\DefineVerbatimEnvironment{CodeOutput}{Verbatim}{}
\newenvironment{CodeChunk}{}{}

% cite package, to clean up citations in the main text. Do not remove.
\usepackage{cite}

\usepackage{color}

% Use doublespacing - comment out for single spacing
%\usepackage{setspace}
%\doublespacing


% % Text layout
% \topmargin 0.0cm
% \oddsidemargin 0.5cm
% \evensidemargin 0.5cm
% \textwidth 16cm
% \textheight 21cm

\title{Postural changes mediate children's visual access to social information}


\author{{\large \bf Alessandro Sanchez} \\ \texttt{sanchez7@stanford.edu} \\ Department of Psychology \\ Stanford University \And {\large \bf Bria Long} \\ \texttt{bria@stanford.edu} \\ Department of Psychology \\ Stanford University
    \And {\large \bf Ally Kraus} \\ \texttt{bria@stanford.edu} \\ Department of Psychology \\ Stanford University
    \And {\large \bf Michael C. Frank} \\ \texttt{mcfrank@stanford.edu} \\ Department of Psychology \\ Stanford University}

\begin{document}

\maketitle

\begin{abstract}
The ability to process social cues--including eye gaze--is a critical
component of children's early language and cognitive development.
However, as children reach their first birthday, they begin to locomote
themselves, walking and exploring their visual environment in an
entirely new way. How do these postural and locomotive changes affect
children's access to the social information relevant for word-learning?
Here, we explore this question by using head-mounted cameras to record
infants' (8-16 months of age) egocentric visual perspective and modern
computer vision algorithms to detect the proportion of faces in infants'
environments. We find that infant's postural changes largely mediate
infants' access to faces, suggesting that these postural and locomotive
developments facilitate infants' emerging linguistic and social
capacities. Broadly, we suggest that the combined use of head-mounted
cameras and the application of novel deep learning algorithms is a
promising avenue for understanding the statistics of infants' visual and
linguistic experience.

\textbf{Keywords:}
social cognition; face-perception; infancy; locomotion; head-cameras;
deep learning
\end{abstract}

\section{Introduction}\label{introduction}

Children are remarkably skilled language learners, connecting arbitrary
labels (``cup'') with specific visual concepts at a rapid pace through
the first two years of life. However, children do not learn words in a
vacuum but in a rich social environment, where social cues provided by
speakers (e.g., eye-gaze) provide strong scaffolding for this learning
process. Indeed, children's ability to effectively process these social
cues may be a key factor in their early language development. For
example,in a longitudinal study, children's level of joint engagement
with their mother was found to predict both their receptive and
productive vocabularies (Carpenter, Nagell, \& Tomasello, 1998). More
recently, 10 month-olds who follow an adult's gaze in an experimental
context have larger vocabularies at 18 months (R. Brooks \& Meltzoff,
2005) and throughout the second year of life {]}Rechele Brooks \&
Meltzoff (2008){]}.

However, as children are learning their first words, their view of the
world is also changing radically (K. Adolph \& Berger, 2007). Infant's
motor abilities improve dramatically near the end of the first year of
life, allowing them to locomote independently and to determine what they
see. These motor changes have drastic consequences for what children
see: crawling and walking infants have radically different views of the
world. During spontaneous play in a laboratory playroom, toddlers are
more likely to look at the floor while crawling than while walking (J.
Franchak, Kretch, Soska, \& Adolph, 2011); walking vs.~crawling infants
tend to have full visual access to their environment and the people in
it (K. S. Kretch, Franchak, \& Adolph, 2014).

One possibility is that these motor improvements have strong
developmental cascades, impacting children's emerging social, cognitive,
and linguistic abilities (Iverson, 2010). Indeed, these postural changes
also impact how children interact with their mothers; walking infants
make different kinds of object-related bids for attention from their
mothers than crawling infants, and tend to hear more action directed
statements (e.g., ``open it'') (Karasik, Tamis-LeMonda, \& Adolph,
2014). More directly, in an observational study, Walle \& Campos (2014)
found that children who were able to walk had both higher receptive and
productive vocabularies. On their account, children's ability to stand
and independently locomote may fundamentally change their ability to
access the social information (e.g., faces, gaze) relative to children
who are still crawling and sitting. In other words, the ability to
access more detailed social information may allow infants to learn words
quicker and more efficiently, facilitating language growth.

Recent work has begun to use egocentric, head-mounted cameras to
document the visual experiences of infants and children --- which even
for walking children are radically different than the adult perspective
(and not easily predicted by our own adult intuitions; (Clerkin, Hart,
Rehg, Yu, \& Smith, 2017; J. Franchak et al., 2011, Yoshida \& Smith
(2008)) ). Children's views tend to be restricted and to be dominated by
objects and hands much more than that of adults (Yoshida \& Smith,
2008), and both computational and empirical work suggest that this
restricted viewpoint may be more effective for learning what objects and
their labels than the comparable adult perspective (Bambach, Crandall,
Smith, \& Yu, 2017; D. Yurovsky, Smith, \& Yu, in press). Further,
recent work also suggests dramatic changes in the child's perspective
over the first two years of life, as views transition from primarily
containing close up view of faces to capturing views of hands paired
with the objects they are acting on (Fausey, Jayaraman, \& Smith, 2016).

Here, we directly ask whether the postural changes that infants
experience as they reach their first birthday change the availability of
the social information relevant for word learning. To do so, we recorded
the visual experience of a group of infants and children using
head-mounted cameras during a brief laboratory free-play session;
children's posture and their orientation to their caregiver were also
recorded from a third-person perspective and hand-annotated. We
capitalize on recent improvements in face detection algorithms (K.
Zhang, Zhang, Li, \& Qiao, 2016, Cao, Simon, Wei, \& Sheikh (2017)) and
analyze the frequences of faces in the child's visual environment.

\section{Methods}\label{methods}

\subsection{Participants}\label{participants}

Our final sample consisted of 36 infants and children, with 8
participants in three age groups: 8 months (6 females), 12 months (7
females), and 16 months (6 females). Participants were recruited from
the surrounding community via state birth records, had no documented
disabilities, and were reported to hear at least 80\% English at home.
Demographics and exclusion rates are given in Table \ref{tab:pop}.

\begin{table}[H]
\centering
\begin{tabular}{rrrrrr}
  \hline
 Group & N & \% incl. & Mean age & Videos length (min) \\ 
  \hline
   8 mo. &   12 & 0.46 & 8.71 & 14.41 \\ 
   12 mo. &  12 & 0.40 & 12.62 & 13.48 \\ 
   16 mo. &  12 & 0.31 & 16.29 & 15.00\\ 
   \hline
\end{tabular}
\caption{\label{tab:pop} Demographics by age group.}
\end{table}

To obtain this final sample, we tested 95 children, excluding 59
children for the following reasons: 20 for technical issues related to
the headcam, 15 for failing to wear the headcam, 10 for fewer than 4
minutes of headcam footage, 5 for having multiple adults present, 5 for
missing CDI data, 2 for missing scene camera footage, 1 for fussiness,
and one excluded for sample symmetry. All inclusion decisions were made
independent of the results of subsequent analyses.

\subsection{Head-mounted camera}\label{head-mounted-camera}

\begin{CodeChunk}
\begin{figure}[H]

{\centering \includegraphics{figs/headcam-1} 

}

\caption[Field of view for three different headcam configurations, with the device we used in the middle]{Field of view for three different headcam configurations, with the device we used in the middle. The lowest camera is pictured for comparison, but was not available until after our study was already in progress.}\label{fig:headcam}
\end{figure}
\end{CodeChunk}

We used a small, head-mounted camera (``headcam'') that was constructed
from a MD80 model camera attached to a soft elastic headband. Videos
captured by the headcam were 720x480 pixels with 25 frames per
second.Detailed instructions for creating this headcam can be found at
\url{http://babieslearninglanguage.blogspot.com/2013/10/how-to-make-babycam.html}.
A fisheye lens was attached to the camera to increase the view angle
from \(32^{\circ}\) horizontal by \(24^{\circ}\) vertical to
\(64^{\circ}\) horizontal by \(46^{\circ}\) vertical (see Figure
\ref{fig:headcam}, left).

Even with the fish-eye lens, the vertical field of view of the camera is
still considerably reduced compared to the child's approximate vertical
field of view, which spans around 100--120\(^{\circ}\) in the vertical
dimension by 6-7 months of age (Cummings, Van Hof-Van Duin, Mayer,
Hansen, \& Fulton, 1988; Mayer, Fulton, \& Cummings, 1988). As we were
primarily interested in the presence of faces in the child's field of
view, we chose to orient the camera upwards to capture the entirety of
the child's upper visual field where the child is likely to see the
faces of adults around them. This allowed us to maximize our chances of
capturing faces that the child would have seen during the play session.

\subsection{Procedure}\label{procedure}

First, all parents signed consent documents in a waiting room where
children were fitted with the headcam. After the child was comfortable
in the waiting room and with the experimenter, the experimenter placed
the headcam on the child's head. If the child was uninterested in
wearing the headcam or tried to take it off, the experimenter presented
engaging toys to try to draw the child's focus away from the headcam
(Yoshida \& Smith, 2008).

After the child was comfortable with wearing the headcam, the child and
caregiver were shown to a playroom for the free-play session--the focus
of the current study. Parents were shown a box containing three pairs of
novel and familiar objects (e.g., a ball and a feather duster, named a
``zem''), and were instructed to play with the object pairs with their
child one at a time, ``as they typically would.'' All parents confirmed
that their child had not previosuly seen the novel toys and were
instructed to use the novel labels to refer to the novel toys.

The experimenter then left the playroom for approximately 15 minutes,
during which a tripod-mounted camera in the corner of the room recorded
the session and the headcam captured video from the child's perspective.

\subsection{Data Processing and
Annotation}\label{data-processing-and-annotation}

\begin{figure*}
\includegraphics[width=6in]{images/framesample.pdf}
\caption{\label{fig:frames} Example face detections made by MTCNN for the headcam videos from a child in each group  (green squares).}
\end{figure*}

Headcam videos were trimmed such that they excluded the instruction
phase when the experimenter was in the room and were automatically
synchronized with the tripod-mounted videos using FinalCut Pro Software.
These sessions yielded videos of 516 minutes (almost a milion frames),
with an average video length of 8.6 minutes (min = 4.53, max = 19.35).

\subsubsection{Posture and Orientation
Annotation}\label{posture-and-orientation-annotation}

We created a set of custom annotations that described the child's
physical posture (e.g.~standing) and the orientation of the caregiver
relative to the child (e.g.~far away). The child's posture was
categorized as being held/carried, prone (crawling or lying), sitting,
or standing. The caregiver's orientation was characterized as being
close to the child, farther from the child, and a global category of
caregiver behind the child. For the first two annotations (close/far
from the child), the caregiver could either be to the the front or to
the side of the child. All annotations were made using OpenSHAPA/Datavyu
software (K. Adolph, Gilmore, Freeman, Sanderson, \& Millman, 2012), and
times when the child was out of view of the tripod camera was marked as
uncodable and was excluded from these annotations.

\subsection{Face Detection}\label{face-detection}

Measuring the availability of caregiver's faces across development was
an important goal of the study. We thus used a total of two face
detection systems to explore this hypothesis and to avoid the cost of
hand-annotating every frame. We capitalized on recent improvements in
computer vision, testing the performance of two algorithms that extract
face information in distinct ways. Results for the evaluation of these
three systems are shown in Table 2.

\subsubsection{Face Detection
Algorithms}\label{face-detection-algorithms}

The first algorithm was based on the work by K. Zhang et al. (2016)
using multi-task cascaded convolutional neural networks (MTCNNs). The
system was built using a novel cascaded CNN-based framework for joint
detection and alignment, built to perform well in real-world
environments where varying illuminations and occlusions are present. We
used a Tensorflow implementation of this algorithm provided by
(\url{https://github.com/davidsandberg/facenet}). Like Viola-Jones, this
detector provided information about the presence of a face as well as
it's size and position.

The second algorithm was a CNN-based pose detector that provided the
locations of 18 body parts (ears, nose, wrists, etc) (Cao et al. (2017),
Simon, Joo, Matthews, \& Sheikh (2017), Wei, Ramakrishna, Kanade, \&
Sheikh (2016)) avaliable at
\url{https://github.com/CMU-Perceptual-Computing-Lab/openpose}. The
system uses a CNN for initial anatomical detection and subsequently
applies part affinity fields (PAFs) for part association, producing a
series of body part candidates. The candidates are then matched to a
single individual and finally assembled into a pose. For the purposes of
our investigation we only made use of the body parts relevant to the
face (ears, eyes, nose). We operationalized face detections as any
frames containing a nose, as any half of a face necessarily contains a
nose.\\

\subsubsection{Detector evaluation}\label{detector-evaluation}

In order to evaluate the performance of these detectors, we constructed
a ``gold set'' of frames by hand labeling both a sample of frames with a
high density of faces (as reported by the detectors) and a random sample
from the remaining frames. This was done so as to not bias our
evaluation by the relatively rare appearance of faces in the dataset. A
face was present in a frame if at least half of the face was showing.
Precision (hits / hits + false alarms), recall (hits / hits + misses),
and F-score (harmonic mean of previous measures) were calculated and are
reported in Table 2. Both detectors performed relatively well on the
gold set, with MTCNN outperforming OpenPose on the random sample but
trailing behind in the high density sample. Due to high performance of
these two detectors we have included the results of both in the analyses
to follow.

\begin{table}[H]
\centering
\begin{tabular}{rrrr}
  \hline
  & P & R & F \\ 
\hline
OpenPose High Density Sample & 0.99 & 0.87 & 0.93 \\ 
OpenPose Random Sample & 0.71 & 0.92 & 0.80 \\
MTCNN High Density Sample & 0.84 & 1.00 & 0.91 \\ 
MTCNN Random Sample & 0.95 & 0.75 & 0.84 \\ 
   \hline
\end{tabular}
\caption{Face detector performance} 
\end{table}

\section{Results}\label{results}

First, we report developmental shifts in infants' posture and their
orientation to their caregiver. Then, we explore how these changes
influence children's visual access to faces across this developmental
time range. Finally, we explore how these changes impact the
accessibility of faces during labeling events.

\subsection{Changes in Posture and
Orientation}\label{changes-in-posture-and-orientation}

\begin{CodeChunk}
\begin{figure}[h]

{\centering \includegraphics{figs/orientation-1} 

}

\caption[Proportion time that infants in each age group spent in each posture (left panel) and orientation relative to their caregiver (right  panel)]{Proportion time that infants in each age group spent in each posture (left panel) and orientation relative to their caregiver (right  panel).}\label{fig:orientation}
\end{figure}
\end{CodeChunk}

We noted characteristic changes in infants' posture and orientation
across this developmental time range. The proportion of time infants
spent sitting decreased with age, and the proportion of time infants
spent standing increased with age. Both 8-month-olds and 12-month-olds
spent equivalent amounts of time either lying/crawling, which was
markedly decreased in the 16-month-olds, who spent most of their time
either sitting or standing (see Figure \ref{fig:posture}). We also
observed characteristic changes in children's orientation relative to
their caregivers: the 8-month-olds spent more time with their caregiver
behind them supporting their sitting positions (see Figure
\ref{fig:orientation}).

\begin{CodeChunk}
\begin{figure}[h]

{\centering \includegraphics{figs/posture-1} 

}

\caption[Proportion time that infants in each age group spent in each posture (left panel) and orientation relative to their caregiver (right  panel)]{Proportion time that infants in each age group spent in each posture (left panel) and orientation relative to their caregiver (right  panel).}\label{fig:posture}
\end{figure}
\end{CodeChunk}

\section{Changes in Access to Faces}\label{changes-in-access-to-faces}

We first examined the proportion of face detections across age; a full
summary can be seen in Figure \ref{fig:detsbyAge}. We observed a slight
U-shaped function both when analyzing the output of the MTCNN and
OpenPose detectors, such that 12-month-olds appeared to experience
slighly fewer faces faces than 8 or 16-month-olds.

Next, we turned to our hypothesis of interest, examining the impact of
postural and locomotive developments on children's visual access to
faces. Overall, we found that children's posture was a major factor both
in how many faces they saw during the play session. Infants who were
sitting saw more faces than infants who were lying down or being
carried, while infants who were standing saw the most faces. We next
examined how the child's orientation relative to their caregiver
impacted their visual access to faces. Overall, we saw that children who
were far away from their caregiver were more likely to see faces than
children who were close to their caregiver; this was true within all age
groups and for all detectors.

\begin{CodeChunk}
\begin{figure*}[h]

{\centering \includegraphics{figs/detByAge-1} 

}

\caption[Proportion face (left) and hand (right) detections as a function of participant's age]{Proportion face (left) and hand (right) detections as a function of participant's age.}\label{fig:detByAge}
\end{figure*}
\end{CodeChunk}

\subsection{Model}\label{model}

To formalize these observations, we fit a generalized logistic
mixed-effect model to all with the presence/absence of a face on every
frame as the dependent variable, and participant's age, orientation, and
posture as predictors. Interactions between predictors were not included
as this maximal model failed to converge. A summary of the coefficients
of this model can be found in Table 3. While age remained a significant
predictor even when accounting for the effects of infants' posture and
caregivers' orientation, it held relatively less predictive power.
Overall, these results confirm that infant's access to faces is heavily
infleunced by their own posture and their caregivers orientation towards
them.

\begin{table}[H]
\centering
\begin{tabular}{rrrrr}
  \hline
 & Estimate & Std. Error & z value & Pr($>$$|$z$|$) \\ 
  \hline
(Intercept) & -4.846 & 0.077 & -62.682 & 0.000 \\ 
  Age & 0.170 & 0.061 & 2.766 & 0.006 \\ 
  Posture-Prone & -0.513 & 0.036 & -14.045 & 0.000 \\ 
  Posture-Sit & 0.648 & 0.034 & 18.853 & 0.000 \\ 
  Posture-Stand & 1.005 & 0.035 & 28.402 & 0.000 \\ 
  Orient.-Close & 1.549 & 0.021 & 72.757 & 0.000 \\ 
  Orient.-Far & 2.258 & 0.022 & 101.138 & 0.000 \\ 
   \hline
\end{tabular}
\caption{Results from GLMM model prediting the presence/absence of a face across the entire dataset (MTCNN detectors).} 
\end{table}

\subsection{Access to Faces During Labeling
Events}\label{access-to-faces-during-labeling-events}

Finally, we analyzed how face detections changed during object labeling
events as a function of infant's posture and orientation. Specfically,
we analyzed a four-second window around each labeling event (e.g.,
``Look at the {[}zem{]}!''); these labeling events were hand-annotated
and automatically synchronized with the frame-by-frame face detections.
We again found that infant's posture impacted the degree to which they
saw their caregivers face during a labeling event; infants who were
sitting or standing were more likely to have access to their caregiver's
face.

\begin{CodeChunk}
\begin{figure}[H]

{\centering \includegraphics{figs/detByNaming-1} 

}

\caption[Proportion face detections as a function of infants's posture]{Proportion face detections as a function of infants's posture}\label{fig:detByNaming}
\end{figure}
\end{CodeChunk}

\section{General Discussion}\label{general-discussion}

We use a head-mounted camera to explore how children's postural and
locomotive development directly impacts their access to social
information relevant for word-learning, here operationalized as the
presence of the faces of their caregiver. We found that children's
posture and orientation towards their caregiver changed systematically
across age, and that all three of these factors dramatically impacted
the proportion of faces that were avaliable in the child's visual field.
Thus, infants postural development is mediating factor that explains
age-related changes in the proportion of faces avaliable in infants'
visual field.

This work also deploys novel advancements in computer vision to the
study of developmental psychology. The field of object detection and
recognition has advanced dramatically in the past five years since the
re-birth of deep learning algorithms Krizhevsky, Sutskever, \& Hinton
(2012), creating a new generation of algorithmic tools. These tools are
both substantially better equipped to deal with noisier, more
complicated datasets and that can extract richer and more detailed
information. Videos from the infant perspective provided sustantial
challenges (e.g., partially occluded faces) for the classc models of
face detedction (e.g., ViolaJones, Viola \& Jones (2004)). Further, as
the headcam technologies employed here were inexpensive
(\textasciitilde{}\$60 a camera) and the computer vision algorithms
freely avaliable, this method is a promising avenue for quanitfying the
visual and social information avaliable to infant learners.

Thus, we suggest that the combined use of these new tools can be
leveraged to understand the changing infant perspecive on the visual
world and the implications of these changes for both linguistic,
cognitive, and social development.

\section{Acknowledgements}\label{acknowledgements}

Thanks to Kaia Simmons, Kathy Woo, Aditi Maliwal, and other members of
the Language and Cognition Lab for help in recruitment, data collection,
and annotation. This research was supported by a John Merck Scholars
grant to MCF. An earlier version of this work was presented to the
Cognitive Science Society in Frank, Simmons, Yurovsky, \& Pusiol (2013).
Please address correspondence to Michael C. Frank, Department of
Psychology, Stanford University, 450 Serra Mall (Jordan Hall), Stanford,
CA, 94305, tel: (650) 724-4003, email: \texttt{mcfrank@stanford.edu}.

\section{References}\label{references}

\setlength{\parindent}{-0.1in} \setlength{\leftskip}{0.125in} \noindent

\hypertarget{refs}{}
\hypertarget{ref-adolph2007}{}
Adolph, K., \& Berger, S. (2007). Motor development. In \emph{Handbook
of child psychology}. Wiley Online Library.

\hypertarget{ref-adolph2012}{}
Adolph, K., Gilmore, R., Freeman, C., Sanderson, P., \& Millman, D.
(2012). Toward open behavioral science. \emph{Psychological Inquiry},
\emph{23}(3), 244--247.

\hypertarget{ref-bambach2017}{}
Bambach, S., Crandall, D. J., Smith, L. B., \& Yu, C. (2017). An
egocentric perspective on active vision and visual object learning in
toddlers. In \emph{Proceedings of the seventh joint ieee conference on
development and learning and on epigenetic robotics}.

\hypertarget{ref-brooks2005}{}
Brooks, R., \& Meltzoff, A. (2005). The development of gaze following
and its relation to language. \emph{Developmental Science}, \emph{8}(6),
535--543.

\hypertarget{ref-brooks2008}{}
Brooks, R., \& Meltzoff, A. N. (2008). Infant gaze following and
pointing predict accelerated vocabulary growth through two years of age:
A longitudinal, growth curve modeling study. \emph{Journal of Child
Language}, \emph{35}(1), 207--220.

\hypertarget{ref-cao2017realtime}{}
Cao, Z., Simon, T., Wei, S.-E., \& Sheikh, Y. (2017). Realtime
multi-person 2D pose estimation using part affinity fields. In
\emph{CVPR}.

\hypertarget{ref-carpenter1998}{}
Carpenter, M., Nagell, K., \& Tomasello, M. (1998). Social cognition,
joint attention, and communicative competence from 9 to 15 months of
age. \emph{Monographs of the Society for Research in Child Development},
\emph{63}(4).

\hypertarget{ref-clerkin2017}{}
Clerkin, E. M., Hart, E., Rehg, J. M., Yu, C., \& Smith, L. B. (2017).
Real-world visual statistics and infants' first-learned object names.
\emph{Phil. Trans. R. Soc. B}, \emph{372}(1711), 20160055.

\hypertarget{ref-cummings1988}{}
Cummings, M., Van Hof-Van Duin, J., Mayer, D., Hansen, R., \& Fulton, A.
(1988). Visual fields of young children. \emph{Behavioural and Brain
Research}, \emph{29}(1), 7--16.

\hypertarget{ref-fausey2016}{}
Fausey, C. M., Jayaraman, S., \& Smith, L. B. (2016). From faces to
hands: Changing visual input in the first two years. \emph{Cognition},
\emph{152}, 101--107.

\hypertarget{ref-franchak2011}{}
Franchak, J., Kretch, K., Soska, K., \& Adolph, K. (2011). Head-mounted
eye tracking: A new method to describe infant looking. \emph{Child
Development}.

\hypertarget{ref-frank2013}{}
Frank, M. C., Simmons, K., Yurovsky, D., \& Pusiol, G. (2013).
Developmental and postural changes in children’s visual access to faces.
In \emph{Proceedings of the 35th annual meeting of the cognitive science
society} (pp. 454--459).

\hypertarget{ref-iverson2010}{}
Iverson, J. M. (2010). Developing language in a developing body: The
relationship between motor development and language development.
\emph{Journal of Child Language}, \emph{37}(2), 229--261.

\hypertarget{ref-karasik2014}{}
Karasik, L. B., Tamis-LeMonda, C. S., \& Adolph, K. E. (2014). Crawling
and walking infants elicit different verbal responses from mothers.
\emph{Developmental Science}, \emph{17}(3), 388--395.

\hypertarget{ref-kretch2014}{}
Kretch, K. S., Franchak, J. M., \& Adolph, K. E. (2014). Crawling and
walking infants see the world differently. \emph{Child Development},
\emph{85}(4), 1503--1518.

\hypertarget{ref-krizhevsky2012imagenet}{}
Krizhevsky, A., Sutskever, I., \& Hinton, G. E. (2012). Imagenet
classification with deep convolutional neural networks. In
\emph{Advances in neural information processing systems} (pp.
1097--1105).

\hypertarget{ref-mayer1988}{}
Mayer, D., Fulton, A., \& Cummings, M. (1988). Visual fields of infants
assessed with a new perimetric technique. \emph{Investigative
Ophthalmology \& Visual Science}, \emph{29}(3), 452--459.

\hypertarget{ref-simon2017hand}{}
Simon, T., Joo, H., Matthews, I., \& Sheikh, Y. (2017). Hand keypoint
detection in single images using multiview bootstrapping. In
\emph{CVPR}.

\hypertarget{ref-viola2004robust}{}
Viola, P., \& Jones, M. J. (2004). Robust real-time face detection.
\emph{International Journal of Computer Vision}, \emph{57}(2), 137--154.

\hypertarget{ref-walle2014}{}
Walle, E. A., \& Campos, J. J. (2014). Infant language development is
related to the acquisition of walking. \emph{Developmental Psychology},
\emph{50}(2), 336.

\hypertarget{ref-wei2016cpm}{}
Wei, S.-E., Ramakrishna, V., Kanade, T., \& Sheikh, Y. (2016).
Convolutional pose machines. In \emph{CVPR}.

\hypertarget{ref-yoshida2008}{}
Yoshida, H., \& Smith, L. (2008). What's in view for toddlers? Using a
head camera to study visual experience. \emph{Infancy}, \emph{13},
229--248.

\hypertarget{ref-yurovsky2012}{}
Yurovsky, D., Smith, L., \& Yu, C. (in press). Statistical word learning
at scale: The baby's view is better. \emph{Developmental Science}.

\hypertarget{ref-zhang2016}{}
Zhang, K., Zhang, Z., Li, Z., \& Qiao, Y. (2016). Joint face detection
and alignment using multitask cascaded convolutional networks.
\emph{IEEE Signal Processing Letters}, \emph{23}(10), 1499--1503.

\end{document}
